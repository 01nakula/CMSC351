\documentclass{article}
\usepackage[utf8]{inputenc}
\usepackage{import}
\usepackage[subpreambles=true]{standalone}
\usepackage{graphicx}
\usepackage{amsmath}
\usepackage{amsfonts}
\usepackage{amssymb}
\usepackage{mathrsfs}
\usepackage{enumerate}
\usepackage{fancyhdr}
\usepackage[colorlinks=true,linkcolor=black,anchorcolor=black,citecolor=black,filecolor=black,menucolor=black,runcolor=black,urlcolor=blue]{hyperref}
\usepackage[margin=1.75in]{geometry}
\usepackage{algorithm, algpseudocode}
\usepackage{tikz-qtree}
\usepackage{ulem}

\tikzstyle{arr}=[fill=white, draw=black, shape=rectangle, scale=.85]
\tikzstyle{cir}=[fill=white, draw=black, shape=circle, scale=.75]

\begin{document}
%%%%%%%%%%%%%%%%%%%%%%%%%%%%
%%     Chapter Content    %%
%%%%%%%%%%%%%%%%%%%%%%%%%%%%
\section{Bubble Sort}
\rule{\textwidth}{1pt}\\
\subsection{Pseudocode}
\begin{algorithm}
\caption{Bubble Sort}
\label{bubble}
\begin{algorithmic}[1]
\For{$i=n$ \textbf{ down to } $2$}
	\For{$j=1$ \textbf{ to} $i-1$}
		\If{$A[j] > A[j+1]$}
			\State $A[j] \leftrightarrow A[j+1]$
		\EndIf
	\EndFor
\EndFor
\end{algorithmic}
\end{algorithm}

\subsection{Analysis of Comparisons}
Bubble Sort is designed in such a manner such that the state of the array to be listed
does \textit{not} change the number of comparisons it makes.
\begin{align*}
	\sum_{i=2}^n \sum_{j=1}^{i-1}1 &= \sum_{i=2}^n i-1 \\
	&= \sum_{i=1}^{n-1} i \\
	&= \frac{(n-1)n}{2} = {n\choose 2}
\end{align*}

\subsection{Analysis of Exchanges}
\subsubsection*{Worst Case}
Worst case is when the list is reverse sorted, there will be the same number of exchanges
as comparisons.
$$\frac{(n-1)n}{2}$$
\subsubsection*{Best Case}
Best case is when the list is already sorted, in which there will be zero exchanges.
\subsubsection*{Average Case}
To find the average case we must count the transpositions (two elements that are out of
order related to one another). In best case there are no transpositions, and in worst case
there are $\frac{(n-1)n}{2}$ transpositions. In a randomly permuated array each element is
equally likely to be out of order so the total number of average case exchanges is half the
comparisons.
$$\frac{1}{2} \cdot \frac{(n-1)n}{2} = \frac{(n-1)n}{4}$$


\end{document}