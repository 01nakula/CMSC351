\documentclass{article}
\usepackage[utf8]{inputenc}
\usepackage{import}
\usepackage[subpreambles=true]{standalone}
\usepackage{graphicx}
\usepackage{amsmath}
\usepackage{amsfonts}
\usepackage{amssymb}
\usepackage{mathrsfs}
\usepackage{enumerate}
\usepackage{fancyhdr}
\usepackage[colorlinks=true,linkcolor=black,anchorcolor=black,citecolor=black,filecolor=black,menucolor=black,runcolor=black,urlcolor=blue]{hyperref}
\usepackage[margin=1.75in]{geometry}
\usepackage{algorithm, algpseudocode}
\usepackage{tikz-qtree}
\usepackage{ulem}

\tikzstyle{arr}=[fill=white, draw=black, shape=rectangle, scale=.85]
\tikzstyle{cir}=[fill=white, draw=black, shape=circle, scale=.75]

\begin{document}
%%%%%%%%%%%%%%%%%%%%%%%%%%%%
%%     Chapter Content    %%
%%%%%%%%%%%%%%%%%%%%%%%%%%%%
\section{Selection Sort}
\rule{\textwidth}{1pt}\\
Suppose we want to select the $k$th smallest elements from a group
of $n$ numbers. What if we want to find the median element?
\subsection{Pseudocode}
\begin{algorithm}
\caption{Selection Sort (\textit{Recursive})}
\begin{algorithmic}[1]
\Function{Select}{$A,k,p,r$}
	\State $s\leftarrow$ \Call{approximate\_median}{$A,p,r$}
	\State $q\leftarrow$ \Call{partition}{$A,p,r,s$}
	\If{$k<q-p+1$}
		\State\Call{Select}{$A,p,q-1,l$}
	\ElsIf{$k > q-p+1$}
		\State\Call{Select}{$A,q+1, r, k-(q-p+1)$}
	\Else
		\State\Return{$q$}
	\EndIf
\EndFunction
\end{algorithmic}
\end{algorithm}
\setcounter{algorithm}{10}
\begin{algorithm}
\caption{Selection Sort (\textit{Non-Recursive})}
\begin{algorithmic}[1]
\Function{Select}{$A,k$}
	\State $p\leftarrow 1$
	\State $r\leftarrow n$
	\Repeat
		\State $s\leftarrow$\Call{approximate\_median}{$A,p,r$}
		\State $q\leftarrow$\Call{partition}{$A,p,r,s$}
		\If{$k<q$} $r\leftarrow q-1$
		\ElsIf{$k>q$} $p\leftarrow q+1$
		\EndIf
	\Until{$k=q$}
	\Return{$q$}
\EndFunction
\end{algorithmic}
\end{algorithm}

\noindent Take an approximate median of the list, partition with this
approximate median ($q$), then look to the left or right
depending on how $k$ compares to the partition.

\subsection{Analysis}
\subsubsection*{Worst Case}
Just like all the other bad quadratic sorting algorithms, selection sort
in the worst case results in $n(n-1)$ comparisons
\subsubsection*{Best Case}
If we assume that our median is the true median or that our approximate median
is close enough we have the reccurence
\begin{align*}
	T(n) &= T\left(\frac{n}{2}\right) + n-1 \hspace{3mm} T(1) = 0\\
	&= (n-1) + \left(\frac{n}{2}-1\right) + \left(\frac{n}{4}-1\right) + \cdots \\
	&\approx 2n
\end{align*}
So best case we have $\approx 2n$ comparisons with the assumption that the
pivot is exactly in the middle.
\subsubsection*{Average Case}
Assume the average case occurs at either the $\frac{1}{4}$ mark or the
$\frac{3}{4}$ mark (similar to quicksort). Under this assumption we can write down
the recurrence (we assume a pessimistic approach in that we always choose the larger side)
\begin{align*}
	T(n) &= T\left(\frac{3n}{4}\right) + n-1 \hspace{3mm} T(1) = 0 \\
	&= (n-1) + \left(\frac{3}{4}n -1\right) + \left(\left(\frac{3}{4}\right)^2 n-1\right) + \cdots \\\
	&= n\left[1+\left(\frac{3}{4}\right)+ \left(\frac{3}{4}\right)^2 + \cdots\right] \\
	&= n\frac{1}{1-\frac{3}{4}} = n\frac{1}{\frac{1}{4}} \\
	&= 4n
\end{align*}
To find a pessimistic view (upper bound, choose the bigger side of the partition) for the average case
we then sum over all possible pivots to get the probabalistic analysis
\begin{align*}
	T(n) &= \sum_{q=1}^n \frac{1}{n}T\left(\text{max}(q-1, n-q)\right) + n-1 \hspace{3mm} T(1)=0 \\
	&=\frac{1}{n}\sum_{q=1}^n T\left(\text{max}(q-1, n-q)\right) + n-1 \\
	&=\frac{2}{n}\sum_{q=\frac{n}{2}}^{n-1} T(q) + n-1 \hspace{5mm}\text{by change of variable}
\end{align*}
Then from here we guess that the algorithm is linear, more specifically we guess
$T(n)\leq an$.
\begin{align*}
	T(n) &= \frac{2}{n}\sum_{q=\frac{n}{2}}^{n-1} aq + n-1\\
	&= \frac{2a}{n}\sum_{q=\frac{n}{2}}^{n-1} q + n-1 \\
	&= \frac{2a}{n} \left[\sum_{q=1}^{n-1}q - \sum_{q=1}^{\frac{n}{2}-1}\right] + n-1 \\
	&= \frac{2a}{n}\left[\frac{n(n-1)}{2} - \frac{\frac{n}{2}\left(\frac{n}{2}-1\right)}{2}\right] + n-1 \\
	&= \frac{2a}{n}\left[\frac{n^2}{2}-\frac{n}{2} - \frac{n^2}{8} + \frac{n}{4}\right] + n-1 \\
	&= an-a-\frac{an}{4} + \frac{a}{2} + n-1 \\
	&= \frac{3}{4}an - \frac{a}{2}+ n-1 \\
	&= \left(\frac{3}{4}a + 1\right)n - \frac{a}{2} - 1 \hspace{5mm}\text{for induction to work we need $\frac{3}{4}a + 1\leq an$} \\
	T(n) &\approx 4n
\end{align*}
\subsection{Finding Median}
For this method we will try finding the explicit median to try to get to
best case scenario from above. 
\begin{enumerate}[1.]
	\item Put the elements into a $5\times \frac{n}{5}$ grid.
	\item Find the median of each column.\hfill $\frac{10n}{5}=2n$ comparisons
	\item Within each column move the small elements in the top, large elements in the bottom and median to the middle.
	\item Find the median of medians.\hfill $T\left(\frac{n}{5}\right)$ comparisons
	\item Move the columns with small medians to the left, large medians to the right, and the median of medians to the middle.
	\item Partition using median of medians as pivot.\hfill $n-1$ comparisons
	\item Recursively call algorithm on proper side.\hfill $T\left(\frac{7n}{10}\right)$ comparisons
\end{enumerate}
Then, all together we have the recurrence
\begin{align*}
	T(n) &\leq 2n+ T\left(\frac{n}{5}\right) + n-1 +T\left(\frac{7n}{10}\right)\hspace{5mm}T(1) =0 \\
	&= T\left(\frac{n}{5}\right) + T\left(\frac{7n}{10}\right) + 3n - 1
\end{align*}
Then from here we guess that the algorithm is linear, more specifically we guess $T(n)\leq an$.
\begin{align*}
	T(n) &= a\frac{n}{5} + a\frac{7n}{10} + 3n-1 \\
	&= a\frac{9}{10}n + 3n - 1 \\
	&= \left(\frac{9}{10}a +3\right)n - 1 \hspace{3mm}\text{for induction to work we need $\frac{9}{10}a +3\leq an$} \\
	T(n) &\approx 30n
\end{align*}
So we see that for $n\geq2^{30}$ finding the median of medians will be beneficial,
but this approach can be extended to a grid of $7\times\frac{n}{7}$ or even $9\times\frac{n}{9}$
which results in increasingly more efficient times.
\end{document}