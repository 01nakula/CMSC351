\documentclass{article}
\usepackage[utf8]{inputenc}
\usepackage{import}
\usepackage[subpreambles=true]{standalone}
\usepackage{graphicx}
\usepackage{amsmath}
\usepackage{amsfonts}
\usepackage{amssymb}
\usepackage{mathrsfs}
\usepackage{enumerate}
\usepackage{fancyhdr}
\usepackage[colorlinks=true,linkcolor=black,anchorcolor=black,citecolor=black,filecolor=black,menucolor=black,runcolor=black,urlcolor=blue]{hyperref}
\usepackage[margin=1.75in]{geometry}
\usepackage{algorithm, algpseudocode}
\usepackage{tikz-qtree}
\usepackage{ulem}

\tikzstyle{arr}=[fill=white, draw=black, shape=rectangle, scale=.85]
\tikzstyle{cir}=[fill=white, draw=black, shape=circle, scale=.75]

\begin{document}
%%%%%%%%%%%%%%%%%%%%%%%%%%%%
%%     Chapter Content    %%
%%%%%%%%%%%%%%%%%%%%%%%%%%%%
\section{Insertion Sort with Sentinel}
\rule{\textwidth}{1pt}\\
\subsection{Pseudocode}
\begin{algorithm}
\caption{Insertion Sort with Sentinel}
\begin{algorithmic}[1]
\State $A[0]\leftarrow -\infty$
\For{$i=2$ \textbf{ to } $n$}
	\State $t\leftarrow A[i]$
	\State $j\leftarrow i-1$
	\While{$t<A[j]$}
		\State $A[j+1]\leftarrow A[j]$
		\State $j\leftarrow j-1$
	\EndWhile
	\State $A[j+1]\leftarrow t$
\EndFor
\end{algorithmic}
\end{algorithm}

\subsection{Analysis of Comparisons}
\subsubsection*{Worst Case}
Worst case is when the array is reverse sorted, and every element must be moved.
The while loop always decrements $j$ to zero to compare against the sentinel value.
$$\sum_{i=2}^n i = \left(\sum_{i=1}^n i\right) -1 = \frac{(n+1)n}{2}-2 = \frac{(n+2)(n-1)}{2}$$
\subsubsection*{Best Case}
Best case is when the array is already sorted, there is only one comparison for each iteration
of the for loop.
$$\sum_{i=2}^n 1 = (n-2) + 1 = n-1$$
\subsubsection*{Average Case}
For average case we have to determine the probability that a given element will move.
So we want the expected value of $\sum_{x\in X} P(x)V(x)$, where $P(x)$ is the probability
that an element will end up a location and $V(x)$ is the number of moves.
\begin{align*}
	\sum_{x\in X} P(x)V(x) &= \sum_{i=2}^n \sum_{j=1}^i \frac{1}{i} \cdot (i-j+1) 
	= \sum_{i=2}^n \frac{1}{i}\sum_{j=1}^i (i-j+1) \\
	&= \sum_{i=2}^n \frac{1}{i}\sum_{j=1}^i j 
	= \sum_{i=2}^n \frac{1}{i} \cdot \frac{(i+1)i}{2} \\
	&= \sum_{i=2}^n \frac{i+1}{2} 
	= \frac{1}{2} \sum_{i=2}^n i+1 \\
	&= \frac{1}{2} \sum_{i=1}^n i-1+(n-1) \\
	&= \frac{1}{2} \left(\frac{(n+1)n}{2} -1 + \frac{(n-1)2}{2}\right) \\
	&= \frac{(n+4)(n-1)}{4}
\end{align*}

\subsection{Analysis of Exchanges}
\subsubsection*{Worst Case}
Worst case is two moves for each iteration of outer loop plus worst case number of comparisons,
minus the time when the comparison is false at the end. A shortcut of this is at each iteration
we do one more move than comparison, so take the value we got above and add the number of loop
iterations.
$$\frac{(n+2)(n-1)}{2} + n$$ 
\subsubsection*{Best Case}
Best case is one initial move in the assignment on line 1 and then two moves during each iteration
of the for loop. So we get,
$$1+ 2(n-1) = 2n -1$$
\subsubsection*{Average Case}
Average case is whenever there is a comparison there is a move except for the single instance
in each iteration of when it evaluates to false. Thus the analysis method is the same as 
worst case.
$$\frac{(n+4)(n-1)}{4} + n$$

\end{document}